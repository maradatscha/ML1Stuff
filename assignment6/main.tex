%!TEX root = main.tex
\documentclass
[
	a4paper,
	11pt,				% 10pt, 11pt oder 12pt - Standard ist 10pt
	pointlessnumbers,	% kein abschlie�enden Punkt hinter den Nummerierungen machen
	%pdftex,
	%chapterprefix,		% Kapitel anschreiben als Kapitel
	%openright, 		
	%openany,
	oneside,
	abstracton,
	final,				%draft,
	%normalheadings,	% �berschriften etwas kleiner (smallheadings)
	%BCOR21mm,			% Bindekorrektur, bspw. 1 cm
	%DIV13,			% f�hrt die Satzspiegelberechnung neu aus
	bibtotoc			% Literaturverzeichniss in toc eintragen
	%bibgerm,			% wegen BibTex in Deutschen Stil, sollte an alle weitergegeben werden
]
{scrartcl}				% KOMA-Script 

%\pagestyle{headings}			% sieht gut aus

% Packages:
%\usepackage[ngerman]{babel}
\usepackage[USenglish]{babel}
\usepackage[T1]{fontenc}		% f�r T1 Zeichensatz
\usepackage{lmodern} 			% Type1-Schriftart f�r nicht-englische Texte

\usepackage{amsmath}
\usepackage{amsthm}
\usepackage{amssymb}
\usepackage{dsfont}				% brauche ich bis jetzt nur f�r die Zahlenbereiche: N,R,Z 
%\usepackage[babel,german=quotes]{csquotes}	% f�r einheitliche Anf�hrungszeichen

\usepackage{tikz}		% syntax layer for pgf, a tae macro package for generating graphics
\usetikzlibrary{plotmarks}
\usepackage{verbatim}

%\usetikzlibrary{chains,decorations}
\usepackage{pgf}
\usepackage{xcolor}				% f�r Farbmischungen
\usepackage{multicol}			% f�r mehrere Spalten
\usepackage{url}	
\usepackage{nameref}			% um Kapitel�berschriften referenzieren zu k�nnen

%\renewcaptionname{ngerman}{\abstractname}{Kurzfassung}

%\usepackage{caption}	% ich mache lieber alles �ber das Koma Script:
	\setcapindent{0em}	% kein Einzug
	\addtokomafont{caption}{\footnotesize} %\footnotesize \small 
	\addtokomafont{captionlabel}{\sffamily\bfseries}

%% Packages f�r Grafiken & Abbildungen %%%%%%%%%%%%%%%%%%%%%%
\usepackage{graphicx} %%Zum Laden von Grafiken
\usepackage{subfig} %%Teilabbildungen in einer Abbildung
\usepackage{float}		% f�r den [H] exakt hier Specifier
\usepackage{pdfpages}

\usepackage{boxedminipage}

% wichtig: hyperef muss als letztes Paket geladen werden, weiter Optionen: [pdfstartview={Fit}, pagebackref]
%\xdefinecolor{urlFarbe}{rgb}{0.0,0.5,0.5}
\usepackage[pdfauthor={Fabian Langguth, Sebastian Koch}, bookmarks=false, pdftex, colorlinks=true, urlcolor=urlFarbe, linkcolor=black, citecolor=black]{hyperref}


\renewcommand{\thefigure}{\arabic{figure}}
\renewcommand{\thetable}{\arabic{table}}


\begin{document}

\begin{center}
    \huge {\bf Assignment 5}
    
    \small Fabian Langguth, Sebastian Koch
    
    Sommersemester 2011
    
    \today
\end{center}

\section*{Cauchy Schwartz Inequality} % (fold)
\label{sec:cauchy_schwartz_inequality}

We want to show that $k(x_1,x_2)^2 \leq k(x_1,x_1)k(x_2,x_2)$. As $k$ is a kernel there exists a $\phi(x)$ s.t. $k(x,x') = \phi(x)^{\top}\phi(x')$. So we can write
\begin{align}
	(\phi(x_1)^{\top}\phi(x_2))^2 &\leq \phi(x_1)^2\phi(x_2)^2 \\
	\bigg(\sum_i \phi(x_1)_i \phi(x_2)_i\bigg)^2 &\leq \sum_i \phi(x_1)^2 \sum_i \phi(x_2)^2
\end{align}
We can now show
\begin{align}
 &	\sum_i \phi(x_1)^2 \sum_i \phi(x_2)^2 - \bigg(\sum_i \phi(x_1)_i \phi(x_2)_i\bigg)^2\\ &= \sum_{i,j} \phi(x_1)_i^2 \phi(x_2)_j^2 -  \bigg(\sum_i \phi(x_1)_i \phi(x_2)_i\bigg)^2 \\
	&= \sum_{i,j} \phi(x_1)_i^2 \phi(x_2)_j^2 -  \sum_{i,j} \phi(x_1)_i \phi(x_2)_i \phi(x_1)_j \phi(x_2)_j \\
	&= \sum_{i,j} \phi(x_1)_i^2 \phi(x_2)_j^2 - \phi(x_1)_i \phi(x_2)_i \phi(x_1)_j \phi(x_2)_j \\
	&= \sum_{i,j} \frac{1}{2}\bigg(\phi(x_1)_i^2 \phi(x_2)_j^2 + \phi(x_1)_j^2 \phi(x_2)_i^2 \bigg) - \phi(x_1)_i \phi(x_2)_i \phi(x_1)_j \phi(x_2)_j \\
	&= \sum_{i,j} \frac{1}{2}\bigg(\phi(x_1)_i^2 \phi(x_2)_j^2 + \phi(x_1)_j^2 \phi(x_2)_i^2 \bigg) - \frac{1}{2}\bigg(2\phi(x_1)_i \phi(x_2)_j \phi(x_1)_j \phi(x_2)_i \bigg)\\
	&= \frac{1}{2}\sum_{i,j} (\phi(x_1)_i^2 \phi(x_2)_j^2 + \phi(x_1)_j^2 \phi(x_2)_i^2 \bigg) - (2\phi(x_1)_i \phi(x_2)_j \phi(x_1)_j \phi(x_2)_i \\
	&= \frac{1}{2} \sum_{i,j} \bigg(\phi(x_1)_i \phi(x_2)_j - \phi(x_1)_j \phi(x_2)_i\bigg)^2\\
	&\geq 0
\end{align}
This directly implies the inequality.

% section cauchy_schwartz_inequality (end)


\end{document}
